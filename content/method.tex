\section{Methodology}

\subsection{Data}
 

The data used was taken from \shortcite{farag-2023data} which uses a mesa stellar evolution code that model the evolution from hydrogen burning all the way up to triple alpha burning stage.Data that was used in training the model are from 0.2 to 3 solar masses. The data is then separated into several section of linearity as shown below to minimise error:
0.2-0.4, 0.5-1.1, 1.2-1.4, 1.5-1.9, 2-2.3, 2.4-3.0


\begin{figure}[H]
	\centering
	\includegraphics[width=\textwidth,height=\textheight]{assets/raw.png}
	\caption{Raw data $0.2-3.0 M\odot$.}
	\label{fig:raw}
\end{figure}

Figure~\ref{fig:raw}
shows the raw data from the evolution code. The data was then cut to only include the hydrogen burning part for the linear regression.

\subsection{Machine Learning}
Sci Kit Learn was used, specifically the Linear Regression module:
\begin{equation}
    y=\beta_{0}+\beta_{1}x_1+\beta_{2}x_2+\beta_{3}x_3+...+\beta_{n}x_n
\end{equation}
where 
$y$ is the independent variables, $\beta_0$ is the intercept, $\beta_{1}, \beta_{2}, \beta_{3},..., \beta_{n}$ is the coefficient \shortcite{epperson-2013}. 

The basis of linear regression is the same as $y= mx+c$ but sklearn used more coefficients to predict the model. Linear regression estimate  coefficient such that the sum of error is very small ~  $10^{-4}$ (the tolerance adopted by sklearn) \shortcite{jolly-2018}. This method is also known as the least square method.

In this project the code will calculate the gradient B (written as J)  between the inputs which are the Masses and the Luminosity and the star ages.
\begin{equation}
    \mqty[
        m_{1}\\m_{2}\\m_{3}  
    ]
    J
    =
    \mqty[
        L_{11}&A_{11}&L_{12}&A_{12}&L_{13}&A_{13}\\
        L_{21}&A_{21}&L_{22}&A_{22}&L_{23}&A_{23}\\
        L_{31}&A_{31}&L_{32}&A_{32}&L_{33}&A_{33}
    ]
    \label{Eq:matrices}
\end{equation}
where $m$ is mass, $L$ is luminosity and $A$ is star age.
Equation\ref{Eq:matrices} is the representation on how the data was arranged in order to do the linear regression method. 

\begin{figure}[H]
    \centering

    \begin{tikzpicture}
        \node (A) at (-4,0) [draw, terminal] {Masses X($n\times 1$ )}; 
        \node (B) at (4,0) [draw, terminal] {Luminosity and Star Age Y($n \times m$)}; 
        \node (C) at (0,-2) [draw, process] {Zeroes Matrix J ($1\times m$)};
        \node (D) at (0,-4) [draw, process] {$\min_{J}||XJ-Y||^{2}$};
        \node (E) at (0,-6) [draw, process] {$J=(X^{T}X)^{-1}X^{T}Y$};
        \node (F) at (0,-8) [draw, process] {$X'J=Y'$};
        \node (G) at (0,-10) [draw, terminal] {OUTPUT};
        \draw[-{Latex}] (A) |- (C);
        \draw[-{Latex}] (B) |- (C);
        \draw[-{Latex}] (C) -- (D);
        \draw[-{Latex}] (D) -- (E);
        \draw[-{Latex}] (E) -- (F);
        \draw[-{Latex}] (F) -- (G);

    \end{tikzpicture}
    \caption{Flowchart of how the code does the linear regression.}
    \label{fig:flow}
\end{figure}

Figure~\ref{fig:flow} shows the flow of how the code does the linear regression. The code will take input Masses in the form of X$(n\times1)$ matrix and Luminosity and Star Age as Y$(n\times m)$ matrix. The code will then generate a zeros matrix J$(1\times m)$. The J will then be calculated using Equation\ref{Eq:leastsq} and then used in prediction.

\begin{equation}
    e=\sqrt{(x_p-x_g)^2+(y_p-y_g)^2}
    \label{Eq:er}
\end{equation}

Then using the Equation~\ref{Eq:er}, the error was calculated in term of the offset between predicted and grid data per mass.

\begin{equation}
    \mathrm{Flux}=\frac{L}{4\pi r^2}
    \label{Eq:fluxcal}
\end{equation}

Using the prediction from the linear regression, we can then use the Formula~\ref{Eq:fluxcal} to calculate for flux of distant low mass star of known mass and distant from earth.
 